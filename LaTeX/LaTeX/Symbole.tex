\chapter*{Nomenklatur}
\addcontentsline{toc}{chapter}{Nomenklatur}

% Tabellen in der Nomenklatur linksbündig ausrichten
\setlength{\LTleft}{0pt}
\setlength{\LTright}{0pt}

\subsection*{Lateinisches Alphabet}
\begin{small}
\begin{flushleft}
\begin{longtable}{@{}ll@{}}
\toprule
\textbf{Variable} & \textbf{Bedeutung} \\
\midrule
\endfirsthead
\toprule
\textbf{Variable} & \textbf{Bedeutung} \\
\midrule
\endhead
\midrule
\multicolumn{2}{r}{\emph{Fortsetzung auf der nächsten Seite}}\\
\endfoot
\bottomrule
\endlastfoot
$A$                 & Fläche; präexponentieller Faktor (Arrhenius) \\
$A_{\mathrm{W}}$    & Wärmeübertragende Wandfläche \\
$a$                 & Reaktionsordnung \\
$c$                 & Konzentration (allgemein) \\
$c_i$               & Konzentration der Komponente $i$ \\
$c_p$               & Isobare Wärmekapazität \\
$E_{\mathrm{A}}$    & Aktivierungsenergie \\
$G$                 & Gibbs-Energie \\
$H$                 & (Reaktions-)Enthalpie \\
$k$                 & Geschwindigkeitskonstante einer Reaktion \\
$K$                 & Gleichgewichtskonstante \\
$L$                 & Charakteristische Länge \\
$m$                 & Masse \\
$\dot m$            & Massenstrom \\
$M$                 & Molare Masse \\
$n$                 & Stoffmenge \\
$\dot n$            & Stoffmengenstrom \\
$p$                 & Druck \\
$Q$                 & Wärme / Wärmemenge \\
$\dot Q$            & Wärmeleistung / Enthalpiestrom \\
$r$                 & Reaktionsrate (volumenspezifisch) \\
$R$                 & Allgemeine Gaskonstante \\
$S$                 & Entropie \\
$t$                 & Zeit \\
$T$                 & Temperatur \\
$T_{\mathrm{in}}$   & Eintrittstemperatur \\
$T_{\mathrm{out}}$  & Austrittstemperatur \\
$u$                 & Strömungsgeschwindigkeit \\
$u_z$               & Axiale Strömungsgeschwindigkeit (PFR) \\
$V$                 & Volumen \\
$\dot V$            & Volumenstrom \\
$x$                 & Stoffmengenanteil / Molenbruch \\
$X$                 & Umsatz / Bilanzkennzahl (z.\,B. CO$_2$-Bilanz) \\
$Y_i$               & Experimentalwert (z.\,B. für MSE) \\
$\hat Y_i$          & Simulationswert (z.\,B. für MSE) \\
$E$                 & Fehlerwert (relativ / mittlere Abweichung) \\
$\mathrm{MSE}$      & Mittlerer quadratischer Fehler \\
$z$                 & Axiale Koordinate \\
\end{longtable}
\end{flushleft}
\end{small}

\subsection*{Griechisches Alphabet}
\begin{small}
\begin{flushleft}
\begin{longtable}{@{}ll@{}}
\toprule
\textbf{Variable} & \textbf{Bedeutung} \\
\midrule
\endfirsthead
\toprule
\textbf{Variable} & \textbf{Bedeutung} \\
\midrule
\endhead
\midrule
\multicolumn{2}{r}{\emph{Fortsetzung auf der nächsten Seite}}\\
\endfoot
\bottomrule
\endlastfoot
$\alpha$            & Wärmeübergangskoeffizient / Reaktionsparameter (kontextabhängig) \\
$\beta$             & Temperaturexponent (Arrhenius) \\
$\Delta$            & Änderung / Differenz (z.\,B. $\Delta H$, $\Delta G$) \\
$\lambda$           & Luftverhältnis; Verhältnis realer zu stöchiometrischer Oxidationszahl \\
$\mu$               & Chemisches Potential \\
$\nu$               & Stöchiometrischer Koeffizient \\
$\phi$              & Äquivalenzverhältnis (Verbrennung) \\
$\rho$              & Dichte \\
$\tau$              & Verweilzeit \\
\end{longtable}
\end{flushleft}
\end{small}

\newpage

\subsection*{Indizes}
\begin{small}
\begin{flushleft}
\begin{longtable}{@{}ll@{}}
\toprule
\textbf{Index} & \textbf{Bedeutung} \\
\midrule
\endfirsthead
\toprule
\textbf{Index} & \textbf{Bedeutung} \\
\midrule
\endhead
\bottomrule
\endlastfoot
$i$                 & Komponente / Speziesindex \\
$j$                 & Reaktionsindex \\
$0$                 & Zulauf / Eintritt \\
$\mathrm{in}$       & Eintritt / Feed \\
$\mathrm{out}$      & Ablauf / Austritt \\
$\mathrm{eq}$       & Gleichgewichtswert \\
$\mathrm{ref}$      & Referenzwert \\
$\mathrm{w}$        & Wand / Umgebung \\
$\mathrm{sim}$      & Simulation \\
$\mathrm{exp}$      & Experiment \\
$\mathrm{noCO2}$    & Referenzfall ohne CO$_2$ \\
$\mathrm{CO2}$      & Fall mit CO$_2$-Zugabe \\
$\mathrm{flame}$    & Flammzone \\
$\mathrm{bypass}$   & Bypass-Zone \\
$\mathrm{recirc}$   & Rezirkulationszone \\
$\mathrm{mix}$      & Mischzone \\
\end{longtable}
\end{flushleft}
\end{small}

\subsection*{Abkürzungen}
\begin{small}
\begin{flushleft}
\begin{longtable}{@{}ll@{}}
\toprule
\textbf{Abkürzung} & \textbf{Bedeutung} \\
\midrule
\endfirsthead
\toprule
\textbf{Abkürzung} & \textbf{Bedeutung} \\
\midrule
\endhead
\bottomrule
\endlastfoot
ATR                 & Autothermal Reforming / reduzierter Inhouse-Mechanismus \\
CFD                 & Computational Fluid Dynamics \\
CRECK               & CRECK Modeling Group Mechanismenfamilie \\
GRI-Mech 3.0        & Gas Research Institute Mechanismus 3.0 \\
NUIGMech 1.1        & Mechanismus der National University of Ireland Galway 1.1 \\
AramcoMech 2.0      & Aramco Reaktionsmechanismus \\
PFR                 & Plug Flow Reactor (Rohrreaktor) \\
PSR                 & Perfectly Stirred Reactor (idealer Rührkesselreaktor) \\
POx                 & Partielle Oxidation (Partial Oxidation) \\
ROM                 & Reduced-Order-Model \\
RWGS                & Reverse Water–Gas Shift \\
WGS                 & Water–Gas Shift \\
CHEMKIN-Pro         & Reaktionskinetik-Software (Ansys) \\
\end{longtable}
\end{flushleft}
\end{small}
