\chapter{Fazit und Ausblick}

    In dieser Arbeit wurde gezeigt, dass die nichtkatalytische Partialoxidation von Erdgas bereits durch einfache, reduzierte Modelle, bestehend aus lediglich zwei idealisierten Reaktoren, präzise abgebildet werden kann. Selbst eine einfache lineare Verknüpfung eines PSR und eines PFRs führte zu Simulationsergebnissen mit geringer Abweichung von den experimentellen Messwerten. 
    Eine noch höhere Genauigkeit der Simulationsergebnisse ließ sich durch das Hinzufügen idealisierter Reaktoren zum bestehenden Netzwerk erreichen, die Strömungsphänomene im Reaktor gesondert abbilden. Dabei zeigte sich jedoch, dass in einem einfachen Reaktor, wie in dieser Arbeit betrachtet, eine grobe Aufteilung des Reaktors besser funktioniert als eine detailreiche Reaktorsegmentierung. Diese detailreiche Segmentierung müsste zuerst durch eine CFD-Analyse abgebildet werden, wodurch sich falsche Annahmen für Rahmenbedingungen in die Simulation des ROMs fortpflanzen würden. Als Folge liefert ein mittelmäßig komplexes Modell den geringsten Fehler. 

    Die oft in der Literatur vorgefundenen Aussagen, dass moderne Reaktionsmechanismen zu vergleichbaren Ergebnissen führen und die Genauigkeiten der genutzten Mechanismen stark von Randbedingungen abhängen, konnten durch eigene Simulationen bestätigt werden. Außerdem konnte die These gestützt werden, dass die Mechanismen für die trockene Reformierung größere Abweichungen liefern als für die herkömmliche POx, da wesentlich bestimmende Reaktionen in diesen Reaktionen nicht ausreichend parametrisiert sind. Dies ist jedoch nicht die alleinige Ursache diese Unterschiede.

    Obwohl für die Entwicklung dieser reduzierten Reaktormodelle eine CFD-Analyse notwendig war, konnte aus den Ergebnissen ein Modell mit hoher Modellgüte entwickelt werden, das mit hoher Effizienz wesentliche Vorgänge präzise abbilden kann. Auf der Grundlage dieses validierten Modells können umfangreiche Parameterstudien durchgeführt werden, und eine Anwendung in der Echtzeitregelung ist denkbar.

    Sollte sich dieser Prozess als Möglichkeit zur Nutzung von Prozessgebundenem Kohlenstoffdioxid im großindustriellen Maßstab beweisen, ist zukünftig mit einem verstärkten Fokus auf die Entwicklung in diesem Bereich zu rechnen, was zur Entwicklung eines für diesen Prozess optimierten Reaktionsmechanismus führen könnte. Auch ist zukünftig die Entwicklung reduzierter Modelle mittels datengestützter Verfahren (z.B. künstliche Intelligenz) denkbar. Obwohl die Rechenleistung immer weiter zunimmt, werden CFD-Simulationen nicht in der Lage sein, komplexe Optimierungsaufgaben solcher Reaktoren durchzuführen. Aus diesem Grund wird die Entwicklung reduzierter Modelle zukünftig einen wichtigen Teil in der Entwicklung neuer Prozesse bzw. Prozessbedingungen spielen. 