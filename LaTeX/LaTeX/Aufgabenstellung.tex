\chapter*{Aufgabenstellung}

Ziel der vorliegenden Studienarbeit ist die Entwicklung und Bewertung reduzierter Reaktormodelle für die nichtkatalytische Partialoxidation von Erdgas. Grundlage bildet die Versuchskampagne GASPOX215 des SCOORE-Projekts (Synthesis gas from recycling of CO$_2$), deren experimentelle Daten als Referenz für die Modellvalidierung dienen. Im Fokus steht dabei die Untersuchung unterschiedlicher Reaktionsmechanismen sowie der Einfluss verschiedener Reaktornetzwerk-Topologien auf die Modellgüte.

Zu Beginn der Arbeit wird eine umfassende Literaturrecherche durchgeführt. Diese umfasst einerseits die theoretischen Grundlagen des Reduced-Order-Modelings (ROM) und dessen Anwendung in der Prozesssimulation, andererseits eine systematische Aufarbeitung der relevanten Reaktionsmechanismen für die Erdgasoxidation. Dadurch wird die Basis geschaffen, um die im weiteren Verlauf eingesetzten Modelle methodisch fundiert auswählen und einordnen zu können.

Im nächsten Schritt erfolgt der Aufbau eines einfachen Reaktornetzwerkes, bestehend aus einem ideal durchmischten Reaktor (PSR) sowie einem Rohrreaktor (PFR). Mit diesem Grundmodell werden die Versuchsbedingungen der Kampagne GASPOX215 des SCOORE Projekts abgebildet \cite{Scoore_Enargus}. Dabei werden Fälle mit und ohne CO\textsubscript{2}-Zugabe betrachtet, um den Einfluss von CO\textsubscript{2}-Importen auf die Reaktionskinetik und die Temperaturprofile zu untersuchen. Die Simulationsergebnisse werden anschließend mit den experimentell erhobenen Daten verglichen und hinsichtlich der Eignung der eingesetzten Reaktionsmechanismen bewertet. Um die Modellqualität zu steigern, wird das Reaktornetzwerk im weiteren Verlauf schrittweise um zusätzliche Reaktoren erweitert. Ziel dieser Erweiterungen ist es, komplexere physikalisch-chemische Vorgänge, wie sie in der realen Versuchsanlage auftreten, adäquat abbilden zu können. Die Auswirkungen der Netzwerkerweiterungen werden systematisch untersucht und hinsichtlich ihres Beitrags zur Verbesserung der Vorhersagequalität analysiert.

Auf Grundlage dieser Erkenntnisse wird schließlich ein optimiertes Reaktornetzwerk entwickelt. Dieses soll eine möglichst realitätsnahe Abbildung der experimentellen Ergebnisse ermöglichen, gleichzeitig jedoch den Anforderungen an ein Reduced-Order-Model in Bezug auf Rechenaufwand und Komplexität genügen. Durch den Vergleich unterschiedlicher Mechanismen und Modellierungsansätze sollen die Grenzen und Potenziale reduzierter Modelle in der Prozesssimulation der nichtkatalytischen Partialoxidation aufgezeigt werden.

Abschließend werden die Vorgehensweise, die erzielten Ergebnisse sowie deren Bewertung in einer schriftlichen Ausarbeitung dokumentiert. Neben der Darstellung der theoretischen Grundlagen und der Beschreibung der Modellierungsansätze umfasst diese eine kritische Diskussion der Simulationsergebnisse im Vergleich zu den experimentellen Daten. Die Arbeit trägt damit zu einem vertieften Verständnis der Möglichkeiten und Grenzen von Reduced-Order-Models in der Auslegung und Analyse von Reaktornetzwerken für die Erdgaspartialoxidation bei.
